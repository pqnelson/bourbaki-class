\documentclass[twoside,usebookdim]{bourbaki}
%\documentclass[twoside]{bourbaki}
\title{Algebra}
\author{Nicholas Bourbaki}
\begin{document}
\frontmatter
\maketitle
\tableofcontents
\mainmatter
\chapter{Algebraic Structures}
\section{Laws of Composition; Associativity; Commutativity}
\subsection{Laws of Composition}

\begin{definition}
Let $E$ be a set. A mapping $f$ of $E\times E$ into $E$ is called a law
of composition on $E$. The value $f(x,y)$ of $f$ for an ordered pair
$(x,y)\in E\times E$ is called the composition of $x$ and $y$ under this
law. A set with a law of composition is called a magma.
\end{definition}

Equation example:
\begin{equation}
E = mc^{2}
\end{equation}

The composition of $x$ and $y$ is usually denoted by writing $x$ and $y$
in a definite order and separating them by a characteristic symbol of
the law in question (a symbol which it may be agreed to omit). Among the
symbols most often used are $+$ and $\cdot$, the usual convection being
to omit the latter if desired; with these symbols the composition of $x$
and $y$ is written respectively as $x + y$ and $x\cdot y$ or $xy$.
A law denoted by the symbol $+$ is usually called addition (the
composition $x + y$ being called the sum of $x$ and $y$) and we say that
it is written additively; a law denoted by the symbol $\cdot$ is usually
called multiplication (the composition $x\cdot y = xy$ being called the
product of $x$ and $y$) and we say that it is written
multiplicatively. In the general arguments of paragraphs 1 to 3 of this
chapter we shall generally use the symbols $\mathbin{\top}$ and $\bot$ to denote
arbitrary laws of composition.

By an abuse of language, a mapping of a subset of $E\times E$ into $E$
is sometimes called a law of composition not everywhere defined on $E$.

Let $(x,y)\mapsto x\mathbin{\top} y$ be a law of composition on a set $E$. Given
any two subsets $X$, $Y$ of $E$, $X \mathbin{\top} Y$ (provided this notation does
not lead to confusion) will denote the set of elements $x\mathbin{\top} y$ in $E$,
where $x\in X$, $y\in Y$ (in other words, the image of $X\times Y$ under
the mapping $(x,y)\mapsto x\mathbin{\top} y$).

If $a\in E$ we usually write $a\mathbin{\top} Y$ instead of $\{a\}\mathbin{\top} Y$ and
$X\mathbin{\top} a$ instead of $X\mathbin{\top}\{a\}$. The mapping $(X, Y)\mapsto X\mathbin{\top} Y$
is a law of composition on the set of subsets of $E$.

Definition. Let $E$ be a magma and $\mathbin{\top}$ denote its law of composition. The law of
composition $(x, y)\mapsto y\mathbin{\top} x$ on $E$ is called the opposite of the
above. The set $E$ with this law is called the opposite magma of $E$.


Definition.
Let $E$ be a set. A mapping $f$ of $E\times E$ into $E$ is called a law
of composition on $E$. The value $f(x,y)$ of $f$ for an ordered pair
$(x,y)\in E\times E$ is called the composition of $x$ and $y$ under this
law. A set with a law of composition is called a magma.

Definition.
Let $E$ be a set. A mapping $f$ of $E\times E$ into $E$ is called a law
of composition on $E$. The value $f(x,y)$ of $f$ for an ordered pair
$(x,y)\in E\times E$ is called the composition of $x$ and $y$ under this
law. A set with a law of composition is called a magma.

The composition of $x$ and $y$ is usually denoted by writing $x$ and $y$
in a definite order and separating them by a characteristic symbol of
the law in question (a symbol which it may be agreed to omit). Among the
symbols most often used are $+$ and $\cdot$, the usual convection being
to omit the latter if desired; with these symbols the composition of $x$
and $y$ is written respectively as $x + y$ and $x\cdot y$ or $xy$.
A law denoted by the symbol $+$ is usually called addition (the
composition $x + y$ being called the sum of $x$ and $y$) and we say that
it is written additively; a law denoted by the symbol $\cdot$ is usually
called multiplication (the composition $x\cdot y = xy$ being called the
product of $x$ and $y$) and we say that it is written
multiplicatively. In the general arguments of paragraphs 1 to 3 of this
chapter we shall generally use the symbols $\mathbin{\top}$ and $\bot$ to denote
arbitrary laws of composition.

By an abuse of language, a mapping of a subset of $E\times E$ into $E$
is sometimes called a law of composition not everywhere defined on $E$.

Let $(x,y)\mapsto x\mathbin{\top} y$ be a law of composition on a set $E$. Given
any two subsets $X$, $Y$ of $E$, $X \mathbin{\top} Y$ (provided this notation does
not lead to confusion) will denote the set of elements $x\mathbin{\top} y$ in $E$,
where $x\in X$, $y\in Y$ (in other words, the image of $X\times Y$ under
the mapping $(x,y)\mapsto x\mathbin{\top} y$).

If $a\in E$ we usually write $a\mathbin{\top} Y$ instead of $\{a\}\mathbin{\top} Y$ and
$X\mathbin{\top} a$ instead of $X\mathbin{\top}\{a\}$. The mapping $(X, Y)\mapsto X\mathbin{\top} Y$
is a law of composition on the set of subsets of $E$.

By an abuse of language, a mapping of a subset of $E\times E$ into $E$
is sometimes called a law of composition not everywhere defined on $E$.

Let $(x,y)\mapsto x\mathbin{\top} y$ be a law of composition on a set $E$. Given
any two subsets $X$, $Y$ of $E$, $X \mathbin{\top} Y$ (provided this notation does
not lead to confusion) will denote the set of elements $x\mathbin{\top} y$ in $E$,
where $x\in X$, $y\in Y$ (in other words, the image of $X\times Y$ under
the mapping $(x,y)\mapsto x\mathbin{\top} y$).

If $a\in E$ we usually write $a\mathbin{\top} Y$ instead of $\{a\}\mathbin{\top} Y$ and
$X\mathbin{\top} a$ instead of $X\mathbin{\top}\{a\}$. The mapping $(X, Y)\mapsto X\mathbin{\top} Y$
is a law of composition on the set of subsets of $E$.


By an abuse of language, a mapping of a subset of $E\times E$ into $E$
is sometimes called a law of composition not everywhere defined on $E$.

Let $(x,y)\mapsto x\mathbin{\top} y$ be a law of composition on a set $E$. Given
any two subsets $X$, $Y$ of $E$, $X \mathbin{\top} Y$ (provided this notation does
not lead to confusion) will denote the set of elements $x\mathbin{\top} y$ in $E$,
where $x\in X$, $y\in Y$ (in other words, the image of $X\times Y$ under
the mapping $(x,y)\mapsto x\mathbin{\top} y$).

If $a\in E$ we usually write $a\mathbin{\top} Y$ instead of $\{a\}\mathbin{\top} Y$ and
$X\mathbin{\top} a$ instead of $X\mathbin{\top}\{a\}$. The mapping $(X, Y)\mapsto X\mathbin{\top} Y$
is a law of composition on the set of subsets of $E$.

\begin{definition}
Let $E$ be a magma and $\mathbin{\top}$ denote its law of composition. The law of
composition $(x, y)\mapsto y\mathbin{\top} x$ on $E$ is called the opposite of the
above. The set $E$ with this law is called the opposite magma of $E$.
\end{definition}

\begin{definition}
Let $E$ be a set. A mapping $f$ of $E\times E$ into $E$ is called a law
of composition on $E$. The value $f(x,y)$ of $f$ for an ordered pair
$(x,y)\in E\times E$ is called the composition of $x$ and $y$ under this
law. A set with a law of composition is called a magma.
\end{definition}

The composition of $x$ and $y$ is usually denoted by writing $x$ and $y$
in a definite order and separating them by a characteristic symbol of
the law in question (a symbol which it may be agreed to omit). Among the
symbols most often used are $+$ and $\cdot$, the usual convection being
to omit the latter if desired; with these symbols the composition of $x$
and $y$ is written respectively as $x + y$ and $x\cdot y$ or $xy$.
A law denoted by the symbol $+$ is usually called addition (the
composition $x + y$ being called the sum of $x$ and $y$) and we say that
it is written additively; a law denoted by the symbol $\cdot$ is usually
called multiplication (the composition $x\cdot y = xy$ being called the
product of $x$ and $y$) and we say that it is written
multiplicatively. In the general arguments of paragraphs 1 to 3 of this
chapter we shall generally use the symbols $\mathbin{\top}$ and $\bot$ to denote
arbitrary laws of composition.

By an abuse of language, a mapping of a subset of $E\times E$ into $E$
is sometimes called a law of composition not everywhere defined on $E$.

Let $(x,y)\mapsto x\mathbin{\top} y$ be a law of composition on a set $E$. Given
any two subsets $X$, $Y$ of $E$, $X \mathbin{\top} Y$ (provided this notation does
not lead to confusion) will denote the set of elements $x\mathbin{\top} y$ in $E$,
where $x\in X$, $y\in Y$ (in other words, the image of $X\times Y$ under
the mapping $(x,y)\mapsto x\mathbin{\top} y$).

If $a\in E$ we usually write $a\mathbin{\top} Y$ instead of $\{a\}\mathbin{\top} Y$ and
$X\mathbin{\top} a$ instead of $X\mathbin{\top}\{a\}$. The mapping $(X, Y)\mapsto X\mathbin{\top} Y$
is a law of composition on the set of subsets of $E$.

\begin{definition}
Let $E$ be a magma and $\mathbin{\top}$ denote its law of composition. The law of
composition $(x, y)\mapsto y\mathbin{\top} x$ on $E$ is called the opposite of the
above. The set $E$ with this law is called the opposite magma of $E$.
\end{definition}

\begin{definition}
Let $E$ be a set. A mapping $f$ of $E\times E$ into $E$ is called a law
of composition on $E$. The value $f(x,y)$ of $f$ for an ordered pair
$(x,y)\in E\times E$ is called the composition of $x$ and $y$ under this
law. A set with a law of composition is called a magma.
\end{definition}

The composition of $x$ and $y$ is usually denoted by writing $x$ and $y$
in a definite order and separating them by a characteristic symbol of
the law in question (a symbol which it may be agreed to omit). Among the
symbols most often used are $+$ and $\cdot$, the usual convection being
to omit the latter if desired; with these symbols the composition of $x$
and $y$ is written respectively as $x + y$ and $x\cdot y$ or $xy$.
A law denoted by the symbol $+$ is usually called addition (the
composition $x + y$ being called the sum of $x$ and $y$) and we say that
it is written additively; a law denoted by the symbol $\cdot$ is usually
called multiplication (the composition $x\cdot y = xy$ being called the
product of $x$ and $y$) and we say that it is written
multiplicatively. In the general arguments of paragraphs 1 to 3 of this
chapter we shall generally use the symbols $\mathbin{\top}$ and $\bot$ to denote
arbitrary laws of composition.

By an abuse of language, a mapping of a subset of $E\times E$ into $E$
is sometimes called a law of composition not everywhere defined on $E$.

Let $(x,y)\mapsto x\mathbin{\top} y$ be a law of composition on a set $E$. Given
any two subsets $X$, $Y$ of $E$, $X \mathbin{\top} Y$ (provided this notation does
not lead to confusion) will denote the set of elements $x\mathbin{\top} y$ in $E$,
where $x\in X$, $y\in Y$ (in other words, the image of $X\times Y$ under
the mapping $(x,y)\mapsto x\mathbin{\top} y$).

If $a\in E$ we usually write $a\mathbin{\top} Y$ instead of $\{a\}\mathbin{\top} Y$ and
$X\mathbin{\top} a$ instead of $X\mathbin{\top}\{a\}$. The mapping $(X, Y)\mapsto X\mathbin{\top} Y$
is a law of composition on the set of subsets of $E$.

\begin{definition}
Let $E$ be a magma and $\mathbin{\top}$ denote its law of composition. The law of
composition $(x, y)\mapsto y\mathbin{\top} x$ on $E$ is called the opposite of the
above. The set $E$ with this law is called the opposite magma of $E$.
\end{definition}

\begin{definition}
Let $E$ be a set. A mapping $f$ of $E\times E$ into $E$ is called a law
of composition on $E$. The value $f(x,y)$ of $f$ for an ordered pair
$(x,y)\in E\times E$ is called the composition of $x$ and $y$ under this
law. A set with a law of composition is called a magma.
\end{definition}

The composition of $x$ and $y$ is usually denoted by writing $x$ and $y$
in a definite order and separating them by a characteristic symbol of
the law in question (a symbol which it may be agreed to omit). Among the
symbols most often used are $+$ and $\cdot$, the usual convection being
to omit the latter if desired; with these symbols the composition of $x$
and $y$ is written respectively as $x + y$ and $x\cdot y$ or $xy$.
A law denoted by the symbol $+$ is usually called addition (the
composition $x + y$ being called the sum of $x$ and $y$) and we say that
it is written additively; a law denoted by the symbol $\cdot$ is usually
called multiplication (the composition $x\cdot y = xy$ being called the
product of $x$ and $y$) and we say that it is written
multiplicatively. In the general arguments of paragraphs 1 to 3 of this
chapter we shall generally use the symbols $\mathbin{\top}$ and $\bot$ to denote
arbitrary laws of composition.

By an abuse of language, a mapping of a subset of $E\times E$ into $E$
is sometimes called a law of composition not everywhere defined on $E$.

Let $(x,y)\mapsto x\mathbin{\top} y$ be a law of composition on a set $E$. Given
any two subsets $X$, $Y$ of $E$, $X \mathbin{\top} Y$ (provided this notation does
not lead to confusion) will denote the set of elements $x\mathbin{\top} y$ in $E$,
where $x\in X$, $y\in Y$ (in other words, the image of $X\times Y$ under
the mapping $(x,y)\mapsto x\mathbin{\top} y$).

If $a\in E$ we usually write $a\mathbin{\top} Y$ instead of $\{a\}\mathbin{\top} Y$ and
$X\mathbin{\top} a$ instead of $X\mathbin{\top}\{a\}$. The mapping $(X, Y)\mapsto X\mathbin{\top} Y$
is a law of composition on the set of subsets of $E$.

\begin{definition}
Let $E$ be a magma and $\mathbin{\top}$ denote its law of composition. The law of
composition $(x, y)\mapsto y\mathbin{\top} x$ on $E$ is called the opposite of the
above. The set $E$ with this law is called the opposite magma of $E$.
\end{definition}

\begin{definition}
Let $E$ be a set. A mapping $f$ of $E\times E$ into $E$ is called a law
of composition on $E$. The value $f(x,y)$ of $f$ for an ordered pair
$(x,y)\in E\times E$ is called the composition of $x$ and $y$ under this
law. A set with a law of composition is called a magma.
\end{definition}

The composition of $x$ and $y$ is usually denoted by writing $x$ and $y$
in a definite order and separating them by a characteristic symbol of
the law in question (a symbol which it may be agreed to omit). Among the
symbols most often used are $+$ and $\cdot$, the usual convection being
to omit the latter if desired; with these symbols the composition of $x$
and $y$ is written respectively as $x + y$ and $x\cdot y$ or $xy$.
A law denoted by the symbol $+$ is usually called addition (the
composition $x + y$ being called the sum of $x$ and $y$) and we say that
it is written additively; a law denoted by the symbol $\cdot$ is usually
called multiplication (the composition $x\cdot y = xy$ being called the
product of $x$ and $y$) and we say that it is written
multiplicatively. In the general arguments of paragraphs 1 to 3 of this
chapter we shall generally use the symbols $\mathbin{\top}$ and $\bot$ to denote
arbitrary laws of composition.

By an abuse of language, a mapping of a subset of $E\times E$ into $E$
is sometimes called a law of composition not everywhere defined on $E$.

Let $(x,y)\mapsto x\mathbin{\top} y$ be a law of composition on a set $E$. Given
any two subsets $X$, $Y$ of $E$, $X \mathbin{\top} Y$ (provided this notation does
not lead to confusion) will denote the set of elements $x\mathbin{\top} y$ in $E$,
where $x\in X$, $y\in Y$ (in other words, the image of $X\times Y$ under
the mapping $(x,y)\mapsto x\mathbin{\top} y$).

If $a\in E$ we usually write $a\mathbin{\top} Y$ instead of $\{a\}\mathbin{\top} Y$ and
$X\mathbin{\top} a$ instead of $X\mathbin{\top}\{a\}$. The mapping $(X, Y)\mapsto X\mathbin{\top} Y$
is a law of composition on the set of subsets of $E$.

\begin{definition}
Let $E$ be a magma and $\mathbin{\top}$ denote its law of composition. The law of
composition $(x, y)\mapsto y\mathbin{\top} x$ on $E$ is called the opposite of the
above. The set $E$ with this law is called the opposite magma of $E$.
\end{definition}



\end{document}
