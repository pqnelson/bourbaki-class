%% \documentclass{article}
%% \usepackage[chapter=5]{bourbaki}
\documentclass[twoside,chapter=5]{bourbaki}
\def\define#1{\textbf{``#1''}}
\title{Example Article}
\author{Alex Nelson}
\date{December 25, 2023}
\begin{document}
\maketitle

\section{Lie Algebras}

\subsection{Commutative Lie Algebras}

\begin{definition}
We call a Lie algebra $\mathfrak{g}$ \define{commutative} if for all
$x,y\in\mathfrak{g}$ we have $[x,y]=0$.
\end{definition}

\begin{remark}
This probably makes sense, since if we view the bracket of a Lie algebra
as its ``multiplication operator'', then we get the results.
\end{remark}

\begin{proposition}
Any module $\mathfrak{m}$ over a commutative ring $R$ may be given a
commutative Lie algebra structure. Morever, this may be uniquely done.
\end{proposition}


\vfill\eject


\begin{definition}
We call a Lie algebra $\mathfrak{g}$ \define{commutative} if for all
$x,y\in\mathfrak{g}$ we have $[x,y]=0$.
\end{definition}

\begin{remark}
This probably makes sense, since if we view the bracket of a Lie algebra
as its ``multiplication operator'', then we get the results.
\end{remark}

\begin{proposition}
Any module $\mathfrak{m}$ over a commutative ring $R$ may be given a
commutative Lie algebra structure. Morever, this may be uniquely done.
\end{proposition}


\vfill\eject


\begin{definition}
We call a Lie algebra $\mathfrak{g}$ \define{commutative} if for all
$x,y\in\mathfrak{g}$ we have $[x,y]=0$.
\end{definition}

\begin{remark}
This probably makes sense, since if we view the bracket of a Lie algebra
as its ``multiplication operator'', then we get the results.
\end{remark}

\begin{proposition}
Any module $\mathfrak{m}$ over a commutative ring $R$ may be given a
commutative Lie algebra structure. Morever, this may be uniquely done.
\end{proposition}


\end{document}
